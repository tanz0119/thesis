%---------------------------------------------------------------------------------
\chapter{Conclusions and Future Directions}
\label{chap:conclu}
%---------------------------------------------------------------------------------
\section{Conclusions}
Hybrid-duplex communications is proposed as a viable step towards enhancing spectrum efficiency for MAV and UAV networks in various realistic environments. Through new analytical frameworks, HBD communications in networks with single or an arbitrary number of uplink and downlink nodes are shown to achieve significantly better performance over HD systems at low SNR regimes.

Beginning with the case of single uplink and downlink nodes in HBD-ACSs and HBD-UCSs, closed-form outage probability, finite SNR diversity gain and finite SNR DMT expressions are obtained for the II, SIC and joint detectors in Rician fading environments. At low SNR regimes, HBD systems are shown to achieve lower outage probability and higher diversity gain than HD systems. Suitable interference scenarios for the II, SIC and joint detectors are also identified. At asymptotic SNR regimes, it is shown that the SIC detector attains zero diversity gain. It is also demonstrated that the diversity gain of the joint detector is not affected by the data rate of the interfering signal under sufficiently strong interference. In contrast, it is established that the data rate of the interference must be lower than the data rate of the desired signal in order for the SIC detector to achieve non-zero diversity gain.

Apart from Rician fading environments, aeronautical communications can also occur over environments where fading and shadowing are jointly present. In this aspect, the outage probabilities of UAV networks with single uplink and downlink UAVs are analyzed for Rician shadowed fading channels. Through new analytical frameworks based on a power series approach, closed-form expressions for the PDF, CDF, fractional moment, and outage probability are obtained under the Rician shadowed fading model. Extensive analysis shows that the impact of shadowing on the SI channel is negligible at the FD-GS. It is also demonstrated that severe shadowing on the desired link of the FD-GS causes higher outage probability, despite SI mitigation measures. At the downlink UAV, it is seen that the joint detector is more robust to shadowing than the II detector.

Analytical frameworks are also proposed for multi-UAV networks with an arbitrary number of uplink and downlink UAVs. Specifically, a BPP-based outage probability analysis framework is proposed for multi-UAV networks with an arbitrary number of uplink UAVs. When compared against HD-UCSs, it is observed that multi-UAV networks employing HBD-UCSs can concurrently support a greater number of uplink UAVs with lower outage probability.

NOMA-aided multi-UAV networks are also investigated as a means to further enhance spectrum efficiency in HBD-based UAV communications. Through the introduction of a BPP-based analytical framework, the ergodic capacity of multi-UAV NOMA-aided FD-HetNets is analyzed over Rician fading channels. It is shown that a higher ergodic capacity is achieved with FD-HetNets than HD-HetNets, with FD-HetNets able to simultaneously support more UAVs on the same spectrum at lower altitudes. In the presence of weak SI suppression and strong oscillator phase noise at the FD-GS, FD-HetNets are still able to achieve higher ergodic sum capacity than HD-HetNets.

As NOMA-aided UAV communications over correlated channels are a possibility when considering multi-antenna transceivers, the performance of NOMA transmissions over correlated Rician shadowed fading channels is investigated for UAVs employing dual-diversity selection combining. Using a bivariate Rician shadowed fading model, the joint PDF, joint CDF, outage probability and finite SNR diversity gain of the NOMA-aided UCS is obtained in closed-form. Performance analysis of the NOMA-aided UCS reveals that a greater number of UAVs can simultaneously share the same spectrum while achieving an outage probability that is comparable to \textcolor{black}{that of} OMA-based UCSs. It is also demonstrated that cross correlation only affects the diversity gain of both NOMA and OMA transmissions at low SNR regimes.

Therefore, the analysis conducted in this thesis demonstrates the feasibility of addressing spectrum scarcity in aeronautical communications through HBD systems.
















%Hybrid-duplex communications is proposed in this thesis as a viable step towards addressing spectrum scarcity in both MAV and UAV networks. To this end, a closed-form analytical framework is first proposed for the performance analysis of HBD-ACSs and HBD-UCSs with single uplink and downlink nodes. Through the proposed analytical framework, outage probability evaluations and finite SNR analysis are conducted for HBD-ACSs and HBD-UCSs in various fading environments employing SIC, II, or JD interference management strategies. The closed-form analytical framework is then extended to account for arbitrary number of uplink and downlink UAVs in large-scale multi-UAV networks. In this aspect, the extended analytical framework employs stochastic geometry tools, i.e., BPPs, to account for the spatial location of UAVs. By employing the BPP-based analytical framework, analysis concerning outage probability, finite SNR diversity gain, and ergodic capacity are conducted for HBD multi-UAV networks with arbitrary number of uplink and downlink UAVs.
%
%Through extensive analysis, HBD-UCSs and HBD-ACSs are shown to attain lower outage probability, higher finite SNR diversity gain, and higher ergodic capacity over commonly encountered fading environments when compared to conventional HD systems. Furthermore, large-scale HBD multi-UAV networks are shown to achieve superior performance over multi-UAV networks with HD-UCSs while simultaneously supporting more uplink and downlink UAVs on the same spectrum. Therefore, the practicality and associated benefits of addressing spectrum scarcity with HBD-ACSs and HBD-UCSs is emphasized in this thesis.

\section{Future Directions}

\subsection{Impact of UAV Flight Velocity}

% Many works in the literature have not considered the impact of UAVs, with some assuming that the Doppler effect can be compensated.
One unique characteristic of UAV communications is the associated flight velocity of UAVs due to mobility. By considering the impact of UAV flight velocity in UAV channel models, a greater understanding of Doppler effects in UAV communications can be obtained. While some works in the literature have ignored UAV flight velocity, e.g., in \cite{azari2018ultra}, the present research and other UAV-related studies have assumed that Doppler can be compensated, e.g., in \cite{lyu2017spectrum} and \cite{zeng2016wireless}.

% Remaining works that considered mobility have taken a few approaches.
From the remaining studies that have considered UAV flight velocity in UAV channel models, a few approaches towards modeling velocity can be seen. In \cite{sharma2017uav,ono2016wireless}, and \cite{yuan2018capacity}, the authors assumed that the UAVs are flying at constant velocities in the UAV channel models. However, such an assumption may not always be valid as UAVs in the same airspace can vary flight velocities in response to air traffic or operational requirements. Another approach is to model flight velocity as a RV, as was done in \cite{pokkunuru2017capacity} where the UAV flight velocity relative to a ground receiver is modeled as a Gaussian RV. Future works can consider extending the approach in \cite{pokkunuru2017capacity} to obtain PDF expressions of Doppler PDF expressions for UAV channel models. 

%\subsection{Regulatory Compliant HBD-UCS}
%Although HBD systems, and other related FD-based systems, have been widely studied in the literature, e.g., \cite{ernest2019outage,tan2018joint,ernest2019power}, developing the HBD-UCS into a practical platform for UAV communications remains an open problem. One of the major consideration behind the implementation of a practical HBD-UCS is to ensure regulatory compliance. 
%
%For instance, under current regulatory limits in Singapore described in \cite{imda_operation2019}, UCSs can only operate within the range of stipulated carrier frequencies and transmit power restrictions. The range of carrier frequencies allocated for UAV communications in Singapore are also shared with other wireless systems. Furthermore, UCSs must not cause interference to other wireless systems operating on the same allocated spectrum. As an example, UCSs are allowed to operate from $2.4 - 2.4835$ GHz \cite{imda_operation2019}. However, it is useful to point out that Bluetooth systems also operate on the same spectrum \cite{imda_spectrum2019}.
%
%To this end, a feasibility analysis must first be conducted before any realistic implementation of an HBD-UCS can occur. Such a feasibility analysis must analyze the impact of uplink (UAV-to-GS) interference, which are generated from UAVs in an HBD-UCS, on other wireless systems on the spectrum. Likewise, the impact of interference on downlink (GS-to-UAV) transmissions in an HBD-UCS, which are caused by other wireless systems, must be examined in detail. The outcome of such an interference analysis will play a key role in determining the reliability of the HBD-UCS, along with other wireless systems on the same spectrum, while adhering to associated transmit power restrictions.

\subsection{FD Transceivers for UAV Communications}
In HBD UAV communications, the spectrum efficiency of UAV communications is improved by enabling uplink and downlink UAVs, equipped with HD transceivers, to concurrently operate on the same spectrum, as shown in \cite{tan2018joint}. To further improve spectrum efficiency, UAVs can be equipped with FD transceivers instead of HD transceivers. Already, research interest in this direction has been seen in recent works, e.g., \cite{zhang2019framework,wang2018spectrum,song2019joint}. 

From a system model perspective, there is little difference in the FD transceiver modeling for UAVs and GSs. For instance, the SI-related terms in (\ref{JD_HBD_UCS_y_gs}) can be introduced in (\ref{JD_HBD_UCS_y_as2}) to model UAV-2 as a FD-capable node in the HBD-UCS. However, in practice, FD transceivers equipped on UAVs must comply with regulatory requirements. For instance, UAVs must adhere to weight restrictions stipulated by the Civil Aviation Authority of Singapore (CAAS) unless it is operating under special permits \cite{caas2019uas}. In this context, the feasibility of having FD transceivers on UAVs that satisfy a small and light form factor requirement remains to be seen. 

Tradeoffs between cost and performance is another challenge in realizing FD transceivers on UAVs. In particular, achieving high levels of SI mitigation in FD transceivers requires high-quality oscillators \cite{korpi2014full}. Similarly, it has been shown in \cite{syrjala2016analysis} that using cheaper and less components leads to considerable nonlinearities in FD transceivers. Evidently, sacrificing performance to reduce the cost of FD transceivers equipped on UAVs may not enable an HBD-UCS to meet UAV communication requirements, e.g., in terms of reliability. As such, the characterization of FD transceiver design limitations in terms of cost and design considerations is an open research problem, which must be addressed before FD-enabled UAVs can be realized.

\subsection{Feasibility of Interference Forwarding in HBD UAV Communications}

The concept of interference forwarding (IF), as an interference management strategy, involves interference being forwarded by a relay node to a destination node for the purpose of interference cancellation \cite{sirigina2016symbol,sirigina2016full}. In particular, the destination node experiences stronger interference due to IF from the relay. As such, SIC is undertaken at the destination node to recover the SOI.

As an interference management strategy, IF enables improved performance at receivers experiencing moderate interference \cite{sirigina2016symbol}. Thus, IF is an attractive alternative to the computationally expensive JD approach in moderate interference scenarios, where interference is neither strong nor weak. In the literature, IF has been studied for both HD or FD relays. In the former, i.e., HD relays, IF occurs over more than one time slot \cite{sirigina2016symbol, chu2018performance}. In contrast, IF with FD relays enables the concurrent reception and transmission of signals on the same spectrum \cite{sirigina2016full,kader2018full}. Therefore, when compared to HD relays, improved spectrum utilization is realizable with FD relays. In this context, IF is a pragmatic interference management technique for HBD UAV communications. When adopted for the HBD-UCS in Fig. \ref{fig:lit_review_interference_management_example}, the FD-enabled GS, or an additional FD-enabled UAV, can act as a relay to perform IF to UAV-2 that is equipped with a SIC detector. 

Although the performance of the HBD-UCS can be improved through IF when moderate interference is experienced at UAV-2, the feasibility of IF for practical HBD UAV communications is an open research problem. For instance, IF for scenarios involving a small number of nodes has been well investigated, e.g., \cite{sirigina2016full, chu2018performance, kader2018full}. However, considering the ubiquitous nature of UAV communications, supporting a considerable number of deployed UAVs is a must for any practical multi-UAV network with HBD communications. In this aspect, the scalability of IF to support a large number of UAVs in an HBD-UCS is currently unknown, with a feasibility study of implementing IF for HBD UAV communications lacking in the existing literature. The outcome of such a feasibility study can, for example, be used to determine the ideal number of FD-enabled GSs or UAVs functioning as relays, and the expected processing delays, for a given number of deployed UAVs. Another outcome of the feasibility study involves quantifying the outage probability and finite SNR diversity gain of IF in HBD UAV communications, which can be compared against the II, SIC, and JD approaches. Such details will, in due time, come in handy for system designers when considering the various available strategies for inter-UAV interference management.

\subsection{SIC-based Detection Complexity and Error Propagation in Power-Domain NOMA}

The signal models and the resultant SINR expressions in Chapter \ref{chap:NOMA_aided_multi_UAV_FD_HetNet} and Chapter \ref{chap:NOMA_bivariate_Rician_Shadowed} enables an arbitrary number of uplink UAVs ($N_U$) and downlink UAVs ($N_D$) to be considered in the multi-UAV HBD-UCS. Yet, when $N_U$ or $N_D$ is large, the SIC detection process can become complex \cite{islam2017power,dai2018survey}. 

For instance, the FD-enabled GS and downlink UAVs perform SIC-based detection in the presence of MUI and inter-UAV interference from uplink UAVs. In order for the SIC-based detection process to be effective, it is necessary that CSI knowledge is made known to the FD-enabled GS and downlink UAVs. Although one can assume that full CSI is available, such an assumption is unrealistic in practice. Instead, a more prudent approach can be to assume that only partial CSI is available at the FD-enabled GS and the downlink UAVs. One possible approach to model partial CSI is to employ the technique in \cite{zlatanov2017capacity}, where channel estimation error is assumed to follow a Gaussian distribution. However, it is important to note that the technique in \cite{zlatanov2017capacity} only reflects the accuracy of the partial CSI acquisition process and not the complexity of partial CSI acquisition. In particular, acquiring the CSI of all UAVs in the multi-UAV HBD-UCS when $N_U$ or $N_D$ is large can increase the complexity of the CSI acquisition process especially when mobility is considered. Thus, accounting for both the accuracy and complexity of partial CSI acquisition in the context of HBD UAV communications is an open research problem.

Apart from SIC-based detection complexity, error propagation is another issue faced in power-domain NOMA for a multi-UAV HBD-UCS when $N_U$ or $N_D$ is increased \cite{islam2017power,dai2018survey}. Specifically, for the SIC-based detection at the FD-enabled GS and downlink UAVs, detection errors in the early stage of SIC detection hinders the detection process of the subsequent stages as interference may not be adequately removed. The issue of error propagation during SIC is further compounded when partial CSI is assumed, since channel estimation errors can contribute to residual interference after SIC \cite{dai2018survey}. In \cite{wang2017sir,zhang2017downlink,kader2018coordinated,kader2018full}, the authors accounted for residual interference after SIC by introducing a residual interference power coefficient. Likewise, $\beta_{x,j}, x \in \{mbs, i\}$ was introduced in Chapter \ref{chap:NOMA_aided_multi_UAV_FD_HetNet} to account for imperfect SIC at the downlink UAVs, where $\beta_{x,j} = 0$ denotes perfect cancellation and $\beta_{x,j} = 1$ denotes no cancellation \cite{kader2018full}. The variable $\beta$ can also be used to represent the accuracy of the CSI acquisition process \cite{wang2017sir}. Thus, by introducing a residual interference power coefficient, e.g., $\beta$, the impact of channel estimation errors on the residual SIC interference can be analyzed. However, it is noted that this approach does not account for error propagation through each SIC stage and instead assumes some level of residual interference at the end of the SIC detection process. Thus, modeling the error propagation for each SIC stage through a comprehensive mathematical framework in power-domain NOMA remains an open research problem.

\subsection{User Pairing in Power-Domain NOMA}
Thus far, the signal models and resultant SINR expressions in Chapter \ref{chap:NOMA_aided_multi_UAV_FD_HetNet} and Chapter \ref{chap:NOMA_bivariate_Rician_Shadowed} are based on the assumption that the uplink and downlink UAVs are assumed to be grouped, i.e., paired, based on distance. Specifically, the motivation behind such an assumption is due to the conjecture that a shorter distance provides stronger channel gains when the average CSI is considered \cite{kader2018full}. However, in practice, the grouping of all UAVs must first be made known to the FD-enabled GS and downlink UAVs to enable the SIC-based detection process to be carried out. Such a problem is known as user pairing in NOMA-related literature. 

In this regard, there exists some studies on user pairing in power-domain NOMA, e.g., \cite{ding2016general,ding2016impact,liang2017user,zhang2017downlink}. In \cite{ding2016general}, users are uniformly distributed in a disc that is divided into two regions. Then, by assuming a distance-based pairing strategy, users in the inner region of the disc are paired with users in the outer region of the disc. Then, the authors studied the impact of power allocation and signal alignment for the distance-based pairing strategy. In \cite{ding2016impact}, the impact of power allocation for a pairing strategy based on channel gain was studied. Specifically, the authors paired a user with weak instantaneous channel gain with a user that has strong instantaneous channel gain before analyzing fixed and dynamic power allocation strategies. Similarly, a user pairing strategy based on instantaneous channel gain was employed in \cite{liang2017user}, where a matching algorithm that optimizes user pairing and transmit power allocation was proposed. The authors in \cite{zhang2017downlink} investigated two strategies for user pairing based on instantaneous channel gain. The first strategy involves randomly choosing two users. Then, the user with the stronger instantaneous channel gain, i.e., strong user, employs SIC-based detection while the other weaker user employs II detection. The second scheme involves comparing the instantaneous channel gains against a pair of threshold variables, $T_1$ and $T_2$, that satisfy $T_2 \leq T_1$. Then, the user with instantaneous channel gain above $T_1$ is designated as the strong user while the other user is designated as the weaker user only if the instantaneous channel gain is below $T_2$.

It is apparent that user pairing strategies based on instantaneous channel gains have been widely investigated. In particular, it is intuitive that user pairing based on instantaneous channel gains can yield better performance in power-domain NOMA transmissions. However, in a multi-UAV HBD-UCS, such approaches may require high overhead costs due to the need for frequent CSI acquisition and UAV scheduling. To avoid these issues, one can consider a distance-based pairing strategy for HBD UAV communications. Such distance-based pairing strategies may be useful in a multi-UAV HBD-UCS when prior knowledge of the UAV trajectories are available. Interestingly, the effectiveness of a distance-based pairing strategy, when compared against instantaneous channel gain-based user pairing, has not been widely studied. Thus, the tradeoffs between both distance-based and channel gain-based user pairing remains an open research challenge for power-domain NOMA in multi-UAV HBD-UCSs. 

%\subsection{Deep-Learning Techniques for Power-Domain NOMA}
%
%The application of deep learning techniques for physical layer wireless communications is a fast growing research area in the literature. One distinct advantage of deep learning techniques is that mathematically tractable models are not required \cite{o2017introduction,qin2019deep}. In contrast, current designs of communication systems, which are driven by mathematical models, may not necessarily provide an accurate analytical description of the underlying physical system, e.g., modeling finite resolution quantization \cite{o2017introduction} and underwater acoustic channels \cite{qin2019deep}. Thus, deep learning techniques are an attractive alternative to traditional model-based designs, with deep learning techniques being proposed to replace either some parts \cite{xiao2017reinforcement} or entire communications systems \cite{kim2018deep}. 
%
%The applicability of deep learning techniques is enabled through deep neural networks (DNNs), where neurons at each layer are trained to recognize patterns which in turn enables DNNs to learn complicated models. Specifically, each neuron takes in a set of inputs $\{x_1, \ldots , x_n\}$ and a set of weights $\{w_1, \ldots , w_n\}$ to produce an output $y=\sigma\big( \sum_{i=1}^n x_i w_i + b_l \big)$, where $\sigma(\bullet)$ is an activation function and $b_l$ denotes a bias at the $l$-th layer \cite{o2017introduction,qin2019deep}. \footnote{A list of activation functions can be found in \cite[Table II]{o2017introduction}.} By replicating the neurons to work as a group, one obtains a layer. Thereafter, DNNs are formed by implementing the layers to work as an assemble \cite{qin2019deep}. To train DNNs, input and output data are labeled during the training process. The purpose of the training process is to update the set $\{b_l, w_1, \ldots , w_n\}$ such that the loss function $L$ is minimized, which can be accomplished through stochastic gradient descent (SGD) \cite{o2017introduction,qin2019deep}. The loss function $L$ describes the difference between the input at the DNN and the final output produced, which can be computed as the mean-squared error or categorical cross-entropy \cite[Table III]{o2017introduction}. Through the layering of neurons, two dominant types of DNNs are noted \cite{qin2019deep}. The first type of DNN is a feedforward neural network (FNN), where each neuron is connected only to adjacent layers. A subtype of FNN, which is known as a deep convolutional network (DCN), is created by ensuring only some neurons are connected to adjacent layers. The second type of DNN is a recurrent neural network (RNN), which takes data from the previous layer and information from the previous timestep as input \cite{qin2019deep,challita2018deep}. Thus, unlike FNNs, RNNs are able to have memory \cite{qin2019deep,challita2018deep}. 
%
%In the literature, a growing body of studies have started to apply deep learning techniques for wireless communication systems. In \cite{challita2018deep}, a RNN framework was proposed for the optimization of UAV trajectory design and resource allocation. Specifically, the authors employed a RNN with feedback connections, i.e., reinforcement learning (RL), to enable UAVs to the optimal trajectory and transmit power allocation. Elsewhere, a DCN-based framework was proposed in \cite{elbir2019cnn} for the joint-design of precoders and combiners for millimeter wave (mmWave) MIMO systems. For NOMA-aided systems, deep learning techniques have also been noted. In \cite{kim2018deep}, a DNN-based framework was proposed for deep learning aided SCMA. In particular, an FNN was employed at the transmitter for the generation of SCMA codebooks that minimizes bit error rates. At the receiver, an FNN was trained to decode the received codewords. The authors demonstrated that the proposed framework achieves lower bit error rate and computation time than conventional SCMA systems. For power-domain NOMA, a RL-based power allocation algorithm was proposed in \cite{xiao2017reinforcement}. Although the proposed algorithm was able to improve the sum rate of users, the performance tradeoff compared to existing model-based power allocation algorithms is currently unknown. 
%
%From the above discussions, deep learning based techniques can offer promising solutions, particularly for power-domain NOMA in multi-UAV HBD-UCS. However, there still exists limited literature on deep learning aided power-domain NOMA systems. For instance, it still remains to be seen how deep learning techniques can be applied for the joint optimization of trajectory design and power allocation in power-domain NOMA for multi-UAV HBD-UCS. Thus, the application of deep learning techniques for multi-UAV HBD-UCS with power-domain NOMA remains an open research area.


